%%%%%%%%%%%%%%%%%%%%%%%%%%%%%%%%%%%%%%%%%
% Plasmati Graduate CV
% LaTeX Template
% Version 1.0 (24/3/13)
%
% This template has been downloaded from:
% http://www.LaTeXTemplates.com
%
% Original author:
% Alessandro Plasmati (alessandro.plasmati@gmail.com)
%
% License:
% CC BY-NC-SA 3.0 (http://creativecommons.org/licenses/by-nc-sa/3.0/)
%
% Important note:
% This template needs to be compiled with XeLaTeX.
% The main document font is called Fontin and can be downloaded for free
% from here: http://www.exljbris.com/fontin.html
%
%%%%%%%%%%%%%%%%%%%%%%%%%%%%%%%%%%%%%%%%%

%----------------------------------------------------------------------------------------
%	PACKAGES AND OTHER DOCUMENT CONFIGURATIONS
%----------------------------------------------------------------------------------------

\documentclass[a4paper,10pt]{article} % Default font size and paper size

\usepackage{fontspec} % For loading fonts
\defaultfontfeatures{Mapping=tex-text}
\setmainfont[SmallCapsFont = Fontin SmallCaps]{Fontin} % Main document font

\usepackage{xunicode,xltxtra,url,parskip} % Formatting packages

\usepackage[usenames,dvipsnames]{xcolor} % Required for specifying custom colors

\usepackage[big]{layaureo} % Margin formatting of the A4 page, an alternative to layaureo can be \usepackage{fullpage}
% To reduce the height of the top margin uncomment: \addtolength{\voffset}{-1.3cm}

\usepackage{hyperref} % Required for adding links	and customizing them
\definecolor{linkcolour}{rgb}{0,0.2,0.6} % Link color
\hypersetup{colorlinks,breaklinks,urlcolor=linkcolour,linkcolor=linkcolour} % Set link colors throughout the document

\usepackage{titlesec} % Used to customize the \section command
\titleformat{\section}{\Large\scshape\raggedright}{}{0em}{}[\titlerule] % Text formatting of sections
\titlespacing{\section}{0pt}{3pt}{3pt} % Spacing around sections
\usepackage{pbox}
\usepackage{multirow}

\begin{document}

\pagestyle{empty} % Removes page numbering

\font\fb=''[cmr10]'' % Change the font of the \LaTeX command under the skills section

%----------------------------------------------------------------------------------------
%	NAME AND CONTACT INFORMATION
%----------------------------------------------------------------------------------------

\par{\centering{\Huge Vicente \textsc{Cunha}}\bigskip\par} % Your name

\section{Dados Pessoais}

\begin{tabular}{rl}
\multirow{2}{*}{\textsc{Endereço:}} & Rua General Lima e Silva, 757, ap 709 \\ &Centro Histórico, Porto Alegre, RS \\
%\textsc{Cidade e Data de Nascimento:} & Passo Fundo, RS  | 27 de novembro de 1993 \\
\textsc{Telefone:} & (51) 8508 0998\\
\textsc{email:} & \href{mailto:vicentecunha1@hotmail.com}{vicentecunha1@hotmail.com}
\end{tabular}

%----------------------------------------------------------------------------------------
%	WORK EXPERIENCE 
%----------------------------------------------------------------------------------------

\section{Experiência Profissional}

\begin{tabular}{p{1.5cm}|p{12cm}}

\pbox{20cm}{\textsc{Dez} 2015 \\ \textsc{Nov} 2014} & Estagiário na \textsc{T\&T (Tools and Technologies)}, Porto Alegre \\
& \textbf{Projetos para Sistemas Embarcados}\\ 
& \footnotesize{Desenvolvimento de \textit{firmware} em C++ para sistemas embarcados Linux e em C para microcontroladores 8051 e AVR. Projeto de \textit{hardware} para sistemas embarcados e \textit{layout} de PCIs com CadSoft EAGLE. Implementação de métodos de processamento de sinais digitais.}\\
\multicolumn{2}{c}{} \\

\end{tabular}

%----------------------------------------------------------------------------------------
%	EDUCATION
%----------------------------------------------------------------------------------------

\section{Educação}

\begin{tabular}{p{1.5cm}|p{12cm}}

\pbox{20cm}{\textsc{Dez} 2015 \\ \textsc{Jan} 2011} & Graduação em Engenharia Elétrica\\
& \textbf{Universidade Federal do Rio Grande do Sul}\\
& \footnotesize{Projeto de Diplomação: ``Métodos de Segmentação Automática de sinais de EMG de Superfície para Classificação de Movimentos Utilizando RNAs''} \\
& \small Orientador: Prof. Alexandre \textsc{Balbinot}\\
\multicolumn{2}{c}{} \\

\end{tabular}

%----------------------------------------------------------------------------------------
%	SCHOLARSHIPS AND ADDITIONAL INFO
%----------------------------------------------------------------------------------------

\section{Bolsas e Certificados}

\begin{tabular}{p{1.5cm}|p{12cm}}

\pbox{20cm}{\textsc{Jul} 2014} & Prêmio Jovem Pesquisador - Ciências Exatas e da Terra \\
&\textbf{XXVI Salão de Iniciação Científica da Ufrgs}\\
& \footnotesize{Trabalho Apresentado: ``Referências de Tensão tipo \textit{sub-bandgap} sem uso de Resistores''}\\
& \small Orientador: Prof. Hamilton \textsc{Klimach}\\
\multicolumn{2}{c}{} \\

\pbox{20cm}{\textsc{Jul} 2014 \\ \textsc{Ago} 2013} & Bolsista REUNI, \textbf{Laboratório de Microeletrônica do PGMICRO-UFRGS}\\
& \footnotesize{Extração de parâmetros de variabilidade dos processos CMOS IBM 130 nm e XFAB 180 nm. Desenvolvimento de \textit{layouts} para circuitos de referência de tensão com Cadence Virtuoso.}\\
\multicolumn{2}{c}{} \\

\textsc{Dez} 2010 & \textit{Cambridge ESOL Level 3 Certificate in ESOL International}\\
& \footnotesize{\textit{Grade A in the Certificate in Advanced English. Performance at Grade A demonstrates an ability at Level 3 of the UK National Qualifications Framework; and Council of Europe Level C2.}}\\
\multicolumn{2}{c}{} \\

\end{tabular}

%----------------------------------------------------------------------------------------
%	INTERESTS AND ACTIVITIES
%----------------------------------------------------------------------------------------

\section{Outras Informações}

\begin{tabular}{p{1.5cm}|p{12cm}}

\pbox{20cm}{2013} & Presidente do Diretório Acadêmico da Engenharia Elétrica (DAELE-UFRGS) \\
& \footnotesize{Atuação como representante discente no colegiado do Departamento de Engenharia Elétrica (DELET) e na Comissão de Graduação de Engenharia Elétrica (COMGRAD-ELE).}\\%Zelo pelo espaço físico do Diretório Acadêmico.}\\
\multicolumn{2}{c}{} \\

\end{tabular}
%----------------------------------------------------------------------------------------

%----------------------------------------------------------------------------------------
%	LANGUAGES
%----------------------------------------------------------------------------------------

\section{Idiomas}

\begin{tabular}{rl}
\textsc{Inglês:} & Avançado \\
\textsc{Português:} & Língua materna\\
\end{tabular}

%----------------------------------------------------------------------------------------
%	COMPUTER SKILLS 
%----------------------------------------------------------------------------------------

\section{Conhecimentos de Informática}

\begin{tabular}{rl}
Conhecimento Intermediário: & C, C++, Git, MATLAB, LaTeX, CadSoft Eagle  \\
Conhecimento Básico: & VHDL, Cadence Virtuoso, NI LabVIEW\\
&\\
& Habituado com sistemas operacionais Linux e Windows\\
\end{tabular}


\end{document}
